\documentclass{article}

\title{Intergration requirments}
\date{2015-02-21}
\author{JeddShneier}

\begin{document}
\pagenumbering{gobble}
\maketitle
\newpage
\pagenumbering{arabic}
As the system will be accesed remotely through the internet it  needs to integrate with web services .The GUI will be integrated online with HTML JavaScript, CSS and any frameworks or libraries the developer decides to use as long as the system itself remains comaptible on all browsers and does not use browser exclusive  functionality. The system must be integrated with the server thus make use of approprate server side technology such s AJAX and PHP. In summary the system needs to be accesed anywhere on any browser through any device regarless of screen resolution, computer operating system or hardware specification. Also the system must work with these technologies in a robust and consitent manner. 
\newline

For mobile developmet the system is required to intergrate with the Andriod mobile operating system. It must work correctly on all previous versions thus no dependcy on current feature sets. It must also be scalable to future Andriod releases and switching from account access through andriod to web access and vice versa must be seemless and secure. On that note the system must intergrate well with security measures.
\newline

Externally the system must be intergrtaed with a database sytem such as LDAP, mongoDB or an appropriate alternative the developer decides on. Provided that the system integrates with the database securely and it remains scalable. The system is also required to send emails to users thus it must have email intergration either through external email servers, or technologies, or an internally created one. Any libraries or packages needed to realise the funtional requirments must be integrated with the other technologies being used.  It was decided that the system does not need to integrate with Google Calenders thus it will not be using the Google API. 
\newline

The system will genertate reports in the form of spreadsheets and this msut be integrated with the appropriate technology such as Microsoft Excel or LibreOffice Calc. The system will also reguraly genertae statistics and graphs so it is imperative the system integrates with statistical modelling technologies. As the system will be primarly an internet based application the  HTTP suit will be primarly used for website data passing from client to server. The sending of emails should adhere to the SMTP suite and message passing through the system can be achieved through TCP. 
\newline

All integratio needs to be seemeless and not intefer with system performance. Security is also a major concern, thus technologies integrated msut not threaten the systems securiy as a whole or other technologies. The system must remain portable to any Andriod device and to any browser. As the system may grow and shrink as users are added and deleted the systems must be scalable. The data stored in the sytem will also increase thus storage integration msut be scalable as well. The system msut remain reliable with each integrated technology added, and any errors must be able to be traced to the technology casuing it.  

\end{document}